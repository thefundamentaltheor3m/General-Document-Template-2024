\section{A Generic Section}

Here is how text looks in this section.

\lipsum[1]

This should tell you how the line spacing, headers, and paragraph spacing look in this template.

\section{Another Section}

I don't usually use this template for lecture notes: I change my \verb|\documentclass| to \verb|book| and usually use \verb|\chapter|s instead of \verb|\section|s.

\subsection{A Subsection}

This is a subsection. These usually appear in a table of contents.

\subsection{Convergence of a Sequence}

We begin by defining a sequence.

\begin{boxdefinition}[Sequence]
    A \textbf{sequence} is a function $f: \N \to \R$.
\end{boxdefinition}

We can define what it means for a sequence to converge to a number $L \in \R$.

\begin{boxdefinition}[Convergence]
    A sequence $f: \N \to \R$ \textbf{converges} to a number $L \in \R$ if for all $\varepsilon > 0$, there exists $N \in \N$ such that for all $n \geq N$, $\abs{f(n) - L} < \varepsilon$.
\end{boxdefinition}

I have custom commands like \verb|\abs{}| and \verb|\parenth{}| for delimiters that resize automatically to the size of the content inside them.

\begin{boxexample}
    \verb|\parenth{\frac{a}{b}}| looks like this:
    \begin{align*}
        \parenth{\frac{a}{b}}
    \end{align*}
\end{boxexample}

You can also use my boxed environments. Find them in \verb|TeX_Setup/environments.tex|.

\begin{boxtheorem}[Bolzano-Weierstrass]
    Every bounded sequence has a convergent subsequence.
\end{boxtheorem}

This is a super cool theorem! First year analysis is goated.

Here's my favourite algebra theorem.

\begin{boxtheorem}[First Isomorphism Theorem]
    Let $G$ and $H$ be groups and let $\varphi: G \to H$ be a group homomorphism. We have an isomorphism
    \begin{align*}
        \quotient{G}{\pker{f}} \cong \pim{f}
    \end{align*}
\end{boxtheorem}

\section{Concluding Remarks}

Play around! Have fun! Look at all my custom shortcuts and environments. Keep the ones you want and get rid of the rest. The world is your oyster!

\section{Generic Text}

\lipsum
